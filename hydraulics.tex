\documentclass[11pt]{article}
\usepackage{amsmath,amssymb}
\usepackage{graphicx}
\graphicspath{ {/Users/kennedy/Desktop} }
\usepackage[margin=1in]{geometry}
\usepackage[parfill]{parskip}  
\title{Simple Plant Hydraulics \large \\ model development and implementation}
\author{Daniel Kennedy - djk2120@columbia.edu \\ Pierre Gentine - pg2328@columbia.edu}


\begin{document}
\maketitle

\section{Model development}

\subsection{Recipe}

1. Solve for maximum stomatal conductance based on the Medlyn model (involves iterating for intercellular CO$_2$).

2. Solve for the vegetation water potential (and associated stomatal conductance) that matches plant water supply with Penman-Monteith demand.

3. Calculate the photosynthesis based on the stomatal conductance solution (involves iterating for intercellular CO$_2$).


\subsection{Plant Water Supply Equations}

\begin{equation}
q = \int_{\psi_{soil}}^{\psi_{leaf}}{\dfrac{K\left(\psi\right)}{z}d\psi}
\end{equation}

\begin{equation}
K(\psi) = \dfrac{\psi - p_2}{p_1 - p_2} \cdot K_{max}
\end{equation}

\begin{equation}
q = \dfrac{K_{max}}{z}\cdot f\left(\psi\right) \cdot \left(\psi_{soil}-\psi_{leaf}-\rho g z\right)
\end{equation}

\begin{equation}
\begin{aligned}
f\left(\psi\right) &= \dfrac{\dfrac{1}{2} \left(\psi_{soil}+\psi_{leaf}\right) - p_2}{p_1 - p_2} \\
0 &\leq f \leq 1
\end{aligned}
\end{equation}

\subsection{Plant Water Demand Equations}

\begin{equation}
g_{c,max} = g_0 + \left(1+\dfrac{g_1}{\sqrt{D}}\right)\dfrac{A}{C_a}
\end{equation}

\begin{equation}
C_i = C_a - \dfrac{A}{g_c}
\end{equation}

\begin{equation}
A = \dfrac{j/4\left(C_i-\Gamma\right)}{C_i+2\Gamma}
\end{equation}

\begin{equation}
g_c = g_{c,max}\cdot h\left(\psi_{leaf}\right)
\end{equation}

\begin{equation}
h\left(\psi_{leaf}\right) = 
\dfrac{1}{
1+\left(\dfrac{\psi_{leaf}}{p_{50}}\right)^a
}
\end{equation}


\begin{equation}
g_w = 1.6g_c
\end{equation}

\begin{equation}
E = \dfrac{\dfrac{\Delta}{\gamma}\left(R_{net}-G\right)+\rho L_v g_a dq}
{L_v\left(1+\dfrac{\Delta}{\gamma}+\dfrac{g_a}{g_w}\right)}
\end{equation}

\begin{equation}
E = q
\end{equation}

\subsection{Bucket}

\begin{equation}
\theta_1 = \theta_0 - \dfrac{q\Delta t}{Z_r}
\end{equation}

\begin{equation}
\psi_{soil}\left(\theta\right) = \psi_{soil,sat}\left(\dfrac{\theta}{\theta{sat}}\right)^{-b}
\end{equation}


\newpage
\section{Why are tall Amazonian forests more resistant to precipitation variability?}

\subsection{Thread 1}

Assume $\dfrac{dk}{d\psi}=0$

\begin{equation}
q = k\Delta{\psi}
\end{equation}

\begin{equation}
\Delta{\psi} = \psi_{soil}-\psi_{leaf}
\end{equation}

\begin{equation}
q = k\left(\psi_{soil}-\psi_{leaf}\right)
\end{equation}

\begin{equation}
\dfrac{dq}{d\psi_{soil}} = k\left(1-\dfrac{d\psi_{leaf}}{d\psi_{soil}} \right)
\end{equation}

\begin{equation}
\sigma = \dfrac{d\psi_{leaf}}{d\psi_{soil}}
\end{equation}

\begin{equation}
\dfrac{dq}{d\psi_{soil}} = k\left(1-\sigma \right)
\end{equation}

Use a reference pressure drop and water flux (e.g. Day 1 of drydown) to infer $k$:
\begin{equation}
\begin{aligned}
k&=\dfrac{q}{\Delta{\psi}} \\
k&=\dfrac{q_0}{\psi_{soil,0}-\psi_{leaf,0}}
\end{aligned}
\end{equation}

Then we have another way to express $\dfrac{dq}{d\psi}$

\begin{equation}
\dfrac{dq}{d\psi_{soil}} = q_0\dfrac{\left(1-\sigma \right)}{\psi_{soil,0}-\psi_{leaf,0}}
\end{equation}

This shows that the loss of water supply depends is related to the absolute potential drop from soil-to-root.
For taller trees, the potential drop from soil-to-root tends to be larger, whereby a given drop in soil potential due to drydown would have a smaller effect as compared to shorter trees.

Likewise the effect of transpiration demand will be larger for taller trees, because for a given increase in transpiration demand ($dq$),
the effect on the potential drop ($\Delta\psi$) will be larger, due to the smaller conductance.

\begin{equation}
\Delta\psi = \dfrac{q}{k}
\end{equation}

\begin{equation}
\dfrac{d\Delta\psi}{dq} = \dfrac{1}{k} = \dfrac{\psi_{soil,0}-\psi_{leaf,0}}{q_0}
\end{equation}



\subsection{Thread 2}
Given the same precipitation sequence, soil potential variability is smaller for taller trees due to deeper roots, 




\end{document}




