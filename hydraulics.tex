\documentclass[11pt]{article}
\usepackage{amsmath,amssymb}
\usepackage{graphicx}
\graphicspath{ {/Users/kennedy/Desktop} }
\usepackage[margin=1in]{geometry}
\usepackage[parfill]{parskip}  
\title{Simple Plant Hydraulics \large \\ model development and implementation}
\author{Daniel Kennedy - djk2120@columbia.edu \\ Pierre Gentine - pg2328@columbia.edu}


\begin{document}
\maketitle

\section{Model development}

\subsection{Recipe}

1. Solve for maximum stomatal conductance based on the Medlyn model (involves iterating for intercellular CO$_2$).

2. Solve for the vegetation water potential (and associated stomatal conductance) that matches plant water supply with Penman-Monteith demand.

3. Calculate the photosynthesis based on the stomatal conductance solution (involves iterating for intercellular CO$_2$).


\subsection{Plant Water Supply Equations}

\begin{equation}
q = \int_{\psi_{soil}}^{\psi_{leaf}}{\dfrac{K\left(\psi\right)}{z}d\psi}
\end{equation}

\begin{equation}
K(\psi) = \dfrac{\psi - p_2}{p_1 - p_2} \cdot K_{max}
\end{equation}

\begin{equation}
q = \dfrac{K_{max}}{z}\cdot f\left(\psi\right) \cdot \left(\psi_{soil}-\psi_{leaf}-\rho g z\right)
\end{equation}

\begin{equation}
\begin{aligned}
f\left(\psi\right) &= \dfrac{\dfrac{1}{2} \left(\psi_{soil}+\psi_{leaf}\right) - p_2}{p_1 - p_2} \\
0 &\leq f \leq 1
\end{aligned}
\end{equation}

\subsection{Plant Water Demand Equations}

\begin{equation}
g_{c,max} = g_0 + \left(1+\dfrac{g_1}{\sqrt{D}}\right)\dfrac{A}{C_a}
\end{equation}

\begin{equation}
C_i = C_a - \dfrac{A}{g_c}
\end{equation}

\begin{equation}
A = \dfrac{j/4\left(C_i-\Gamma\right)}{C_i+2\Gamma}
\end{equation}

\begin{equation}
g_c = g_{c,max}\cdot h\left(\psi_{leaf}\right)
\end{equation}

\begin{equation}
h\left(\psi_{leaf}\right) = 
\dfrac{1}{
1+\left(\dfrac{\psi_{leaf}}{p_{50}}\right)^a
}
\end{equation}


\begin{equation}
g_w = 1.6g_c
\end{equation}

\begin{equation}
E = \dfrac{\dfrac{\Delta}{\gamma}\left(R_{net}-G\right)+\rho L_v g_a dq}
{L_v\left(1+\dfrac{\Delta}{\gamma}+\dfrac{g_a}{g_w}\right)}
\end{equation}

\begin{equation}
E = q
\end{equation}


\newpage
\section{Why are tall Amazonian forests more resistant to precipitation variability?}

\subsection{Thread 1}
Potential drop from soil-to-root is larger in tall trees. \\
Variability in potential drop due caused by variability in VPD and APAR is larger in tall trees. \\
As such the same variability in soil potential may ``feel'' smaller to taller trees \\

$q$ and $k$ are the leaf area basis transpiration and hydraulic conductance respectively. \\
Assume that tall trees and short trees have the same $q$.

\begin{equation}
\begin{aligned}
q &= k_{tall}\Delta\psi_{tall}  \\
q &= k_{short}\Delta\psi_{short}
\end{aligned}
\end{equation}

\begin{equation}
\begin{aligned}
 k_{tall}    &= \dfrac{q}{\Delta\psi_{tall}}  \\
 k_{short} &= \dfrac{q}{\Delta\psi_{short}}
\end{aligned}
\end{equation}

From the literature (need citation):

\begin{equation}
\Delta\psi_{tall}>\Delta\psi_{short}
\end{equation}

Therefore, we infer that:

\begin{equation}
k_{tall}<k_{short}
\end{equation}

This makes sense because conductances scale inversely with length. 
The tall trees' longer conducting path is more resistive than shorter trees'.

Now define $\delta\psi$ as the potential drop needed to satisfy an extra $\delta q$ of transpiration demand.

\begin{equation}
\delta q = k \delta\psi
\end{equation}

Given that $k_{tall}<k_{short}$, for a given $\delta q$

\begin{equation}
\delta\psi_{tall} > \delta\psi_{short}
\end{equation}

As a result, the variability in $\Delta\psi_{tall}$ will be larger than $\Delta\psi_{short}$
given then same variability in transpiration demand (e.g. due to changes in VPD).

\begin{equation}
\psi_{leaf} = \psi_{soil} - \Delta\psi
\end{equation}

The variability in $\psi_{leaf}$ can be partitioned into the variability in $\psi_{soil}$ and the variability in $\Delta\psi$. 
If the variability in $\Delta\psi$ is larger for tall trees compared to short trees, maybe the variability in $\psi_{soil}$ ``feels'' smaller,
because it makes up a smaller portion of the total variability in $\psi_{leaf}$.
This could partly explain why tall Amazonian forests are more resistant to precipitation variability and more susceptible to VPD.

\subsection{Thread 2}
Given the same precipitation sequence, soil potential variability is smaller for taller trees due to deeper roots, 




\end{document}




